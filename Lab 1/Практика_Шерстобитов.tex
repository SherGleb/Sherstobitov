% !TeX spellcheck = ru_RU
\documentclass[twoside,12pt]{article}
\usepackage{titlesec}
\usepackage{enumitem}
\usepackage[T2A,T1]{fontenc}
\usepackage[utf8]{inputenc}
\usepackage[russian]{babel}
\usepackage{amsmath,amssymb,amsthm}
\usepackage{fancyhdr}
\pagestyle{fancy}
\fancyhf{}
\fancyfoot{}
\renewcommand{\headrulewidth}{0pt}
\setlength{\headheight}{15pt}
\usepackage[a4paper,bindingoffset=0.2in,%
left=1in,right=1in,top=1in,bottom=1in,%
footskip=.10in]{geometry}
%table
\newcommand{\eps}{\varepsilon}
\newcommand{\inprod}[1]{\left\langle #1 \right\rangle}
\newcommand{\handout}[5]{
	\noindent
	\begin{center}
		\framebox{
			\vbox{
				\hbox to 5.78in { {\bf Научно-исследовательская практика} \hfill #2 }
				\vspace{4mm}
				\hbox to 5.78in { {\Large \hfill #5  \hfill} }
				\vspace{2mm}
				\hbox to 5.78in { {\em #3 \hfill #4} }
			}
		}
	\end{center}
	\vspace*{4mm}
} 

\newcommand{\lecture}[4]{\handout{#1}{#2}{#3}{#4}{Cистема верстки LaTeX  #1}}
\begin{document}
\lecture{}{Лето 2020}{}{Шерстобитов Глеб}	
\newpage	
\renewcommand{\headrulewidth}{0pt}
\lhead{\textbf{\thepage}}
\setcounter{page}{331}
\chead{\textbf{Уравнение Пелля}}
\rhead{\textbf{Глава 13}}
\noindent Из-за ошибочного представления о том, что рациональные и не обязательно целочисленные значения были допупстимы, не было трудностей в предоставлении ответа; он просто разделил отношение
\[ (r^2 + d)^2 - d(2r)^2 = (r^2 - d)^2\]
\noindent на величину $(r^2 - d)^2 $, чтобы прийти к решению 
\[x = \frac{r^2 + d}{r^2 - d} , \; y = \frac{2r}{r^2 - d}, \]
\noindent где $r \neq d$ - произвольное рациональное число. Этот ответ, разумеется, был отвергнут Ферма, который написал, что решение в виде дробей, найденных среди самых простых элементов арифметики, его не удовлетворяют. Теперь, проинформированные обо всех условиях задачи Браункер и Уоллис, совместно разработали предварительный метод для решения уравнения $x^2 - dy^2 = 1 $ в целых числах, при этом не имея возможности предоставить доказательство того, что он всегда будет работать. По всей видимости, почести достались Браункеру, поскольку Уоллис с некоторой гордостью похвалил его за то, что тот "сохранил незапятнанную славу" англичан, победивших в прежние времена французов.\\
\indent Помимо вышесказанного, следует отметить, что целенаправленные усилия Ферма по созданию новой традиции в арифметике с помощью математического поединка были в основном неудачными. За исключением Френикля, которому не хватало способности соперничать с Ферма, теория чисел не была особо привлекательна ни для одного из его современников. Предмет не изучался математиками до тех пор, пока Эйлер, по прошествии почти столетия, не продолжил с того же места, где остановился Ферма. Эйлер и Лагранж внесли свой вклад в решение знаменитой проблемы 1657 года. Преобразовав $ \sqrt{d}$ в бесконечную цепную дробь, Эйлер (1759 г.) изобрел алгоритм для получения наименьшего целого решения $x^2 - dy^2 = 1$, но ему не удалось показать, что процесс приводит к решению, отличному от $x=1,\; y=0$. В этом вопросе необходимо было разобраться Лагранжу. Поэтому. завершая теоретические материалы, оставленные незавершенными Эйлером, Лагранж в 1768 г. опубликовал первое строгое доказательство того, что все решения возникают в результате непрерывного расширения дроби. Это объясняет, почему алгоритм приводит к решению, отличному от $x=1, \; y=0 $.\\
\indent В результате ошибочного упоминания, центральная точка разногласия, уравнение $x^2 - dy^2 =1 $, вошла в литературу под названием "Уравнение Пелля". Это было результатом приписывания Эйлером решения английскому математику Джону Пеллу (1611-1685 гг.), который имел мало общего с этой проблемой. Эйлер, должно быть, перепутал их вклад в беглом чтении Opera Mathematica Уоллиса \\(1693 г.), в котором излагается метод решения задачи Броункера, а также информация о работе Пелла по диафантовому анализу. По всем правилам мы должны называть уравнение $x^2 - dy^2 = 1$ "Уравнение Ферма". И хотя историческая ошибка давно признана, имя Пелла - это то, что неизменно связано с уравнением.\\
\indent В независимости от целочисленного значения d, уравнение $x^2 - dy^2 = 1 $ решается при $x = \pm 1,\; y = 0 $. Если $d < -1$, то $x^2 - dy^2 \geq 1 $ (за исключением случая $x = y = 0 $); когда $d = -1 $, появляются еще два решения, а именно $x=0,\; y = \pm 1 $. Случай, когда $d$ является квадратом, легко отбрасывается.  Ибо если  $d = n^2$ для некоторого $n$, то уравнение $x^2 - dy^2 = 1$ можно записать в виде 
$$(x + ny)(x-ny)=1 $$ 
\noindent который возможен тогда и только тогда, когда $x+ny=x-ny=\pm 1$; Из этого следует, что 
$$x=\frac{(x + ny) + (x - ny) }{2} =\pm 1$$
\noindent и уравнение не имеет решений кроме тривиальных $x = \pm 1, \; y = 0$.\\
\indent Отныне мы будем ограничивать наше исследование уравнения Пелля до единственной интересной ситуации, когда $d$ - положительное челое число, а не квадрат. Допустим, что решение $x \; y$ этого уравнения является положительным, когда $x$ и $y$ положительны. Поскольку за искючением $y=0$ решения находятся среди комбинаций знаков $\pm x, \; \pm y $, понятно, что все решения будут известны после нахождения положительных. По этой причине и ищутся только положительные решения уравнения $x^2 - dy^2 = 1$.\\
\indent Результат говорит нам о том, что любую пару натуральных чисел, удовлетворяющих уравнению Пелля, можно получить из цепной дроби, представляющей иррациональное число $\sqrt d$.\\ \\
\noindent \textsc{Теорема 13-14. }
\textit{Пусть $p,\; q$ - положительное решение уравнения $x^2 - dy^2 = 1$. Тогда $\frac{p}{q}$ является подходящей дробью $\sqrt{d} $} \\ \\
\noindent \textsc{\textit{Доказательство:}} По условию $p^2 - dq^2 = 1$. Тогда имеем 
$$(p-q\sqrt{d})(p+q\sqrt{d}) $$ \\
\noindent подразумевая, что $p>q\sqrt{d}$, а также что
$$ \frac{p}{q} - \sqrt{d} = \frac{1}{q(p + q\sqrt{d})}.$$\\
В результате получаем
$$ 0 < \frac{p}{q} - \sqrt{d} < \frac{\sqrt{d}}{q(q\sqrt{d} +q\sqrt{d})} = \frac{\sqrt{d}}{2q^2\sqrt{d}} = \frac{1}{2q^2}.$$\\
\noindent Теорема показывает, что $\frac{p}{q}$ является подходящей дробью $\sqrt{d} $ \\
\indent В общем случае, обратное утверждение предыдущей теоремы неверно: не все подходящие $\frac{p_n}{q_n}$ являются решением $x^2 - dy^2 = 1$. Тем не менее, можно сказать о величине значений, принимаемых последовательностью $p_n^2 - dq_n^2.$\\ \\
\noindent \textsc{Теорема 13-15.}\textit{ Если отношение $p$ к $q$ сходится к непрерывной дроби $\sqrt{d}$, то $x=p, \: y=p$ является решением одного из уравнений }
$$x^2 - dy^2 = k, $$
\noindent где $|k| < 1 + 2\sqrt{d}.$ \\ \\
\noindent \textsc{\textit{Доказательство:}} Если $\frac{p}{q}$ сходится к $\sqrt{d}$, то следствие из теоремы 13-11 гарантирует, что 
$$ |\sqrt{d} - \frac{p}{q}| < \frac{1}{q^2} $$ \\
\noindent и, следовательно, 
$$|p - q\sqrt{d}| < \frac{1}{q}. $$ \\
\noindent Поэтому получаем
$$|p + q\sqrt{d}| = |(p -q\sqrt{d}) + 2q\sqrt{d}| < \frac{1}{q} + 2q\sqrt{d}  < (1 + 2\sqrt{d})q$$ \\
\noindent Два эти неравенства объединяются в следующее:
$$|p^2 - dq^2| = |p - q\sqrt{d}| |p + q\sqrt{d} | < \frac{1}{q}(1 + 2\sqrt{d})q = 1 + 2\sqrt{d}, $$ \\
\noindent Что и требовалось доказать. \\
\indent В качесве примера рассмотри случай, когда $d = 7$. Используя непрерывные дроби, получим $\sqrt{7} = [2;\overline{1,1,1,4}]$. Первые несколько конвергентов $\sqrt{7}$ определяются следующим образом:
$$\frac{2}{1}, \; \frac{3}{1}, \; \frac{5}{2}, \; \frac{8}{3}, \; ... \;.$$ \\
\noindent Пробегая вычисления $p_n^2 - 7q_n^2 ,$ мы обнаружим, что
$$2^2 - 7\cdot 1^2 = -3, \; 3^2 - 7\cdot 1^2 = 2, \; 5^2 - 7 \cdot 2^2 = -3, \; 8^2 - 7\cdot 3^2 =1, $$ 
\noindent откуда $ x = 8, \: y = 3$ дает положительное решение уравнения $x^2 - 7y^2 = 1.$ \\
\indent Несмотря на то, что можно проводить довольно подробное исследование периодических непрерывных дробей, намерения подробно изучать эту область в книге нет. В рассмотренных выше примерах читатель может продолжить разложение дробной части $\sqrt{d}$ в виде
$$\sqrt{d} = [a_0;\overline{a_1,a_2,\ldots,a_n}];$$ 
\noindent Таким образом, периодическая часть начинается после первого слагаемого $\sqrt{d}$. Также верно, что последний член периода $a_n$ всегда равен $2a_0$, и что период, за исключением последнего члена является симметричным (симметрическая часть может иметь или не иметь средний член). Это типично для общего случая.
Не вдаваясь в детали доказательства, следует отметить, что если $d$ - положительное целое число, не являющееся идеальным квадратом, то продолжение разложения дроби $\sqrt{d}$ обязательно имеет вид
$$\sqrt{d} = [a_0; \overline{a_1, a_2, a_3, \ldots , a_3,a_2,a_1,2a_0}] $$
\noindent В случае, когда $d = 73$, например, разложение имеет вид
$$\sqrt{19} = [4; \overline{2,1,3,1,2,8}]. $$
\noindent В то время как $d = 73$ дает
$$\sqrt{73} = [8;\overline{1,1,5,5,1,1,16}].$$
\noindent Среди всех $d<100$, самый длинный период имеет число $\sqrt{94}$, а именно 16 членов:
$$\sqrt{94} = [9;\overline{1,2,3,1,1,5,1,8,1,5,1,1,3,2,1,18}].$$
\indent В прилагаемой таблице перечислены разложения в цепные дроби для $\sqrt{d}$, где d -это целое число от 2 до 40, не являющееся квадратом.\\ \\
\begin{center}
\begin{tabular}{ l l l }
	$\sqrt{2} = [1;\overline{2}]$ & $\sqrt{17} = [4;\overline{8}]$ & $\sqrt{29} = [5;\overline{2,1,1,2,10}]$ \\
	$\sqrt{3} = [1;\overline{1,2}]$ & $\sqrt{18} = [4;\overline{4,8}]$ & $\sqrt{30} = [5;\overline{2,10}]$\\
	$\sqrt{5} = [2;\overline{4}]$ & $\sqrt{19} = [4;\overline{2,1,3,1,2,8}]$ & $\sqrt{31} = [5;\overline{1,1,3,5,3,1,1,10}]$ \\
	$\sqrt{6} = [2;\overline{2,4}]$ & $\sqrt{20} = [4;\overline{2,8}]$ & $\sqrt{32} = [5;\overline{1,1,1,10}]$ \\
	$\sqrt{7} = [2;\overline{1,1,1,4}]$ & $\sqrt{21} = [4;\overline{1,3,1,8}]$ & $\sqrt{33} = [5;\overline{1,2,1,10}]$\\
	$\sqrt{8} = [2;\overline{1,4}]$ & $\sqrt{22} = [4;\overline{1,2,4,2,1,8}]$ & $\sqrt{34} = [5;\overline{1,4,1,10}]$ \\
	$\sqrt{10} = [3;\overline{6}]$ & $\sqrt{23} = [4;\overline{1,3,1,8}]$ & $\sqrt{35} = [5;\overline{1,10}]$\\
	$\sqrt{11} = [3;\overline{3,6}]$ & $\sqrt{24} = [4;\overline{1,8}]$ & $\sqrt{37} = [6;\overline{12}]$ \\
	$\sqrt{12} = [3;\overline{2,6}]$ & $\sqrt{26} = [5;\overline{10}]$ & $\sqrt{38} = [6;\overline{6,12}]$ \\
	$\sqrt{13} = [3;\overline{1,1,1,6}]$ & $\sqrt{27} = [5;\overline{5,10}]$ & $\sqrt{39} = [6;\overline{4;12}]$ \\
	$\sqrt{14} = [3;\overline{1,2,1,6}]$ & $\sqrt{28} = [5;\overline{3,2,3,10}]$ & $\sqrt{40} = [6;\overline{3,12}]$ \\
	$\sqrt{15} = [3;\overline{1,6}]$
\end{tabular}
\end{center}
\newpage
\begin{thebibliography}{3}
	\bibitem{Sulsky1994}
	Львовский С.М. - Набор и верстка в системе LATEX. 5-е изд. 2014.
	\bibitem{LiuLiu}
	Burton D. M. - Elementary number theory. 1980.
\end{thebibliography}
	\end{document}

