\documentclass[aspectratio=169]{beamer}
\usepackage[utf8]{inputenc}

\usepackage[main=russian,english]{babel}
\usetheme{Montpellier}
\usecolortheme{seahorse}

\title{Угрозы безопасности и защита облачных решений и технологий}
\author{Шерстобитов Глеб ИГоревич
2 курс  \\ Учебная практика} 
\date{\ 3 июля 2020г.}
\institute{Балтийский федеральный университет им. Канта}


\begin{document}
	
	\begin{frame}
		\titlepage 
	\end{frame}
	
	\begin{frame}
	\textbf{Задача:} Выявить угрозы безопасности облачных систем и определить наиболее эффективные методы защиты. \\ 
	
	\textbf{Методы решения:}
	\begin{itemize}
		\item Обеспечение безопасности сведений путем их шифрования
		\item Использование туннеля
		виртуальной частной сети (VPN), связывающего клиента и сервер для получения публичных облачных услуг, который обеспечивает безопасное соединение
		\item Аутентификация - защита паролем. С Целью предоставления наиболее
		значительной надежности, зачастую прибегают к таким средствам, как токены и
		сертификаты.
		\item Изолирование пользователей с помощью персональной виртуальной машины и виртуальной сети
	\end{itemize}	
	\end{frame}

	\begin{frame}
	\textbf{Результаты:} По завершению курсовой работы были изучены основные понятия и принципы работы облачных технологий, их главные уязвимости, а также сформирована стратегия минимализации ущерба при разных типах атак.
	\end{frame}
	
\end{document}